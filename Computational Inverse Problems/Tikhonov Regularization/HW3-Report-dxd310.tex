\documentclass{article}

\usepackage[pctex32]{graphics}
\usepackage{amssymb}
\usepackage[left=3cm,top=2cm,right=3cm,nohead,nofoot]{geometry}
\usepackage{latexsym}
\usepackage{epsfig}
\parindent0pt
\parskip6pt
\pagestyle{empty}
%\pagestyle{empty}
%\usepackage{fancybox}
\usepackage{pstcol}
\newcommand{\R}{{\mathbb R}}
\title{Homework 1}
\author{Firstname Lastname}
\begin{document}
\maketitle

This is a small template for homework. 

\bigskip
{\Large {\bf Problem 1}}
\bigskip

We write a Matlab code for computing
\[
 b = {\mathsf A}x,\quad x\sim{\mathcal N}(0,\sigma^2{\mathsf I}_3),
\]
where $\sigma = 0.1$. This is how I include a snippet of my Matlab code:

\begin{verbatim}
A     = [1,2,3, ...
         4,5,6];
sigma = 0.1;
x     = sigma*randn(3,1);   % Random input
b     = A*x;
\end{verbatim}

\bigskip
{\Large {\bf Problem 2}}
\bigskip

This problem requires an inclusion of a figure that I produced using Matlab, see Figure~\ref{fig:my figure}. The figure files are saved in eps-format in the same folder where this text document is. Here is how I include the plot.

\begin{figure}[h!]
\centerline{
\includegraphics[width= 6cm]{LeftPanel.eps}\includegraphics[width= 6cm]{RightPanel.eps}
}
\caption{\label{fig:my figure}The left panel represents the left panel of the figure, while, curiously, the right panel is the right panel. How about that?}
\end{figure}

and then the story continues.

\end{document}